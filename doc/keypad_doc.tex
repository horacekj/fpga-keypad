\documentclass[12pt,a4paper]{article}
\usepackage[unicode]{hyperref}
\usepackage[czech]{babel}
\usepackage[utf8]{inputenc}
\usepackage[T1]{fontenc}
\usepackage{lmodern}
\usepackage{graphicx}
\textwidth 16cm \textheight 24.6cm
\topmargin -1.3cm 
\oddsidemargin 0cm
\usepackage{url}
\usepackage{listings}

\begin{document}

\setcounter{page}{1}  % nastaví čítač stránek znovu od jedné
\pagenumbering{arabic} % číslování arabskymi číslicemi

\title{\normalsize Projekt předmětu PB170 \\ \huge Čtečka klávesnice KeyPad \\ \normalsize semestr podzim 2016, FI MUNI}
\author{Jan Horáček\\
	445326}
\date{\today}
\maketitle

%%%%%%%%%%%%%%%%%%%%%%%%%%%%%%%%%%%%%%%%%%%%%%%%%%%%%%%%%%%
\section{Popis projektu}

Tématem projektu je načítání stavu tlačítkové klávesnice F-KV16KEY\footnote{\href{https://www.gme.cz/f-kv16key-black}{https://www.gme.cz/f-kv16key-black}}, resp. libovolné jiné klávesnice
splňující standard, kterým komunikuje tato klávesnice. K vizualizaci načítaných hodnot jsou využity
segmentové displeje.

Projekt je vyvíjen v prostředí Quartus II 13.0 pro vývojovou desku Altera DE2.

%%%%%%%%%%%%%%%%%%%%%%%%%%%%%%%%%%%%%%%%%%%%%%%%%%%%%%%%%%%
\section{Zapojení hardware}

Keypad je připojen na GPIO konektor \texttt{GPIO II}, konkrétně sloupce jsou připojeny na piny 1-4 jako vstupy
a řádky na piny 5-8 jako výstupy. Sloupce musí být opatřeny pull-up rezistory.

Žádný další hardware není nutný.

%%%%%%%%%%%%%%%%%%%%%%%%%%%%%%%%%%%%%%%%%%%%%%%%%%%%%%%%%%%
\section{Popis funkce}

Přiložená implementace načítá stav klávesnice pomocí skenování přes
jednotlivé řádky a~postupné čtení vstupních pinů. Každý přečtený znak
(mimo znaků \texttt{*} a \texttt{\#}) je zapsán na nejpravější segment
segmentového displeje vestavěného do Altera II. Již zobrazené znaky
jsou posunuty o jeden segment doleva. Znaky \texttt{*} a \texttt{\#}
způsobí vymazání všech displejů.

FPGA by mělo bez problému zvládat stisk více kláves zároveň.

%%%%%%%%%%%%%%%%%%%%%%%%%%%%%%%%%%%%%%%%%%%%%%%%%%%%%%%%%%%
\section{Implementace}

Přiložený soubor \texttt{keyboard.v} obsahuje implementaci kořenového
modulu, který řeší veškerou logiku, soubor \texttt{divider.v} obsahuje
implementaci děličky, která je využívána při skenování klávesnice.

Modul \texttt{keyboard} instanciuje modul děličky tak, aby skenoval
klávesnici s frekvencí $1\ kHz$ v always bloku. V always bloku je vždy
provedeno uzemnění jednoho řádku, v~další iteraci přečtení vstupů a v další
iteraci nastavení výstupu řádku do stavu vysoké impedance. Tento postup se
opakuje pro každý řádek.

V případě, že FPGA detekuje stisk nové klávesy, nastaví flag \texttt{new\_key}, který je využit jako trigger v druhém always bloku --
bloku, který při stisku nové klávesy provede přenos na display.

Druhý always blok nejdříve posune všechny znaky o jeden segment doleva a~následně podle indexu přečteného znaku provede jeho zobrazení na
nejpravější displej.

%%%%%%%%%%%%%%%%%%%%%%%%%%%%%%%%%%%%%%%%%%%%%%%%%%%%%%%%%%%
\section{Diskuze}

Přiložený kód vypadá celkem přímočaře, přesto mi na něm přijde několik věcí
zajímavých (a možná zbytečně složitých).

\begin{itemize}
	\item Při návrhu skenu kláves jsem se snažil nejdříve navrhnout synchonní always blok, který by v jedné iteraci přečetl celý řádek. Tedy by nejdříve zapnul výstup řádku, pak přečetl vstupy sloupců a následně opět vypnul výstup řádku. To se mi ale vůbec nedařilo. A nejhorší na tom je, že vůbec nevím proč. Zkoušel čekat mezi nastavením výstupu řádku a čtením vstupu s vírou v prodlevu způsobenou propagací signálu, ale to nemělo vůbec žádný efekt. Proto jsem se nakonec rozhodl načítání implementovat takto \uv{trojstupňově}.
	\item Využití \texttt{new\_key} jako triggeru always bloku mi přijde nanejvýš hezké.
	\item Překvapilo mě, že jsem vůbec nemusel řešit zákmity na vstupech. Nikdy se mi nestalo, že bych stiskem klávesy způsobil posun o víc, než jeden segment.
	\item Opravdu si musím pro účely změny výstupů v always bloku vytvářet lokální proměnné \texttt{reg}, do kterých až mohu přiřazovat? U nastavování segmentových displejů vidím spoustu řádků kódu, které by tam potenciálně nemusely být.
\end{itemize}

\end{document}